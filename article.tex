% Method
%   Analysis of data || Processing
%    Difficulties related to tagging
%  Uncertainty
%
% Results
%   Tables
%   Graphs
% Conclusion















% Make two column format for LaTex 2e
\documentclass[10pt,twocolumn,a4paper]{article}

% Use the following packages
\usepackage[utf8]{inputenc}     % Support for UTF-8
\usepackage{dcolumn}            % Align table columns on decimal point
\usepackage{graphicx}           % Include graphics
\usepackage{asymptote}          % Vector graphics language
\usepackage{listings}           % Pretty source code listings
\usepackage{url}                % Handle URLs better

% Set dimensions of columns, gap between columns, and paragraph indent
\setlength{\textheight}{8.875in}
\setlength{\textwidth}{6.875in}
\setlength{\columnsep}{0.3125in}
\setlength{\topmargin}{0in}
\setlength{\headheight}{0in}
\setlength{\headsep}{0in}
\setlength{\parindent}{1pc}
\setlength{\oddsidemargin}{-.1875in}  % Centers text
\setlength{\evensidemargin}{-.1875in}

% Add the period after section numbers, and adjust spacing
\newcommand{\Section}[1]{\vspace{-8pt}\section{\hskip -1em.~~#1}\vspace{-3pt}}
\newcommand{\SubSection}[1]{\vspace{-3pt}\subsection{\hskip -1em.~~#1}
\vspace{-3pt}}


\begin{asydef}
// Global Asymptote definitions can be put here.
import three;
usepackage("bm");
texpreamble("\def\V#1{\bm{#1}}");
// One can globally override the default toolbar settings here:
// settings.toolbar=true;
\end{asydef}





\begin{document}

% Make the title bold
\title{\bf The Left Arm Bias in ITF Taekwon-Do Patterns \\ Chon-Ji to Juche}

% Author information
\author{Bjørn Michelsen, III. Dan \\
  \texttt{<bjorn@bmichelsen.no>} \\
  \\
  ITF Tromsdalen School of Taekwon-Do \\
  Tromsø, Norway
  }

% Print date
\date{\today}

% Produce the title
\maketitle


\section*{\centering Abstract}
\begin{em}

%An academic abstract typically outlines four elements germane to the completed work:

    %* The research focus (i.e. statement of the problem(s)/research issue(s) addressed);
    %* The research methods used (experimental research, case studies, questionnaires, etc.);
    %* The results/findings of the research; and
    %* The main conclusions and recommendations

  Todo.
\end{em}





\Section{Introduction}

  A \emph{pattern} in International Taekwon-Do Federation (ITF) is defined as
  a choreographed sequence of fundamental movements. These \emph{fundamental
  movements} are basic elements which represent an attack or defense against a
  specific target area, or a predetermined action of an attacker.

  Combined in a pattern, it allows the student to systematically deal with one
  or several imaginary opponents under various assumptions, using every
  available attacking and blocking tool from different directions.
  Furthermore, the techniques should also be evenly distributed between the
  left and the right side\cite{cyclo:vol1}.

  It has, however, recently been observed that a bias towards the right leg
  exist. The study by Gibbs\cite{rlb} show that there is a total of 152 kicks
  in patterns Chon-Ji to Tong-Il. 85 kicks are with the right leg, and 67
  kicks are with the left leg. This means that 56\% of the kicks are with the
  right leg, and 44\% with the left leg.

  The work by Gibbs focused on kicks in all patterns, whereas this article
  examines patterns Chon-Ji to Juche. Looking at kicks in said patterns using
  Gibbs' paper, we find that there are 72 kicks in total. 42 kicks are with
  the right leg, and 30 kicks are with the left leg. Roughly 58\% of the kicks
  are with the right leg, and about 42\% with the left. Thus, the right leg
  bias still holds.

  In this article, therefore, we analyze hand techniques in patterns Chon-Ji
  to Juche using tags and report our findings of a left arm bias.


\SubSection{Tags}

  In a hierarchical and exclusive classification system, each object belong in
  one unambiguous category, which in turn is within another more general one.
  Take the hierarchy of folders in a computer file system, for example. If we
  want to organize articles on birds native to Africa in a directory, we have
  several options:
  
  \begin{lstlisting}
    1. ~/articles/birds
    2. ~/articles/africa
    3. ~/articles/africa/birds
    4. ~/articles/birds/africa
  \end{lstlisting}

  How we choose to organize the folders reflects a decision concerning the
  relative importance of each characteristic\cite{golder:ct}. Folder names and
  levels are in themselves informative, in that, like tags, they describe the
  information held within them\cite{jones:folders}.

  The directory structure of (1), and (2) make central the fact that the
  folders are about ``birds'' and ``africa''. But, they don't tell anything
  about the other category. In (3) and (4) the files are organized by both
  categories, and show that there is a parent-child or sibling relationships
  between the folder names\cite{mathes:folk}.

  \emph{Tags}, on the other hand, are non-hierarchical descriptive terms,
  keywords or labels that is attached to an object for later retrieval
  \cite{golder:ct}\cite{huang:tt}\cite{shirky:ontology}\cite{wikipedia:tags}.

  They are \emph{non-hierarchical} and \emph{inclusive}, which is to say that
  an object can be associated with a great variety of tags
  simultaneously\cite{golder:ct}. The \emph{descriptive terms},
  \emph{keywords} or \emph{labels} are metadata connected to a given object
  which describe a concept or type of information\cite{heymann:ccch}.


\SubSection{Tagging}

  Tagging is the act of organizing a collection of objects into related
  groups\cite{shirky:ontology}.




% Why do we tag?
%   this kind of metadata helps describe an item
%   allows it to be found again by searching
%   future navigation, filtering or search
%   later retrieval
%   categorize content using keywords
%   grouping related objects
%   classifies
%   organize
%   for the purpose of classification and retrieval
%   display venn diagram
%   explain venn diagram
% How do we tag (tagging)?
%   define tagging


% end the section with something like..
% tags add metadata to an object
% tagging groups objects
% the reason for why we bother is because we want to find the objects again




\SubSection{Retrieval}


% explain how later retrieval is done by using the venn diagram


% Venn diagram
\def\A{A}
\def\B{\V{B}}

%\begin{figure}
\begin{center}
\begin{asy}
size(4cm,0);
pen colour1=red;
pen colour2=green;

pair z0=(0,0);
pair z1=(-1,0);
pair z2=(1,0);
real r=1.5;
path c1=circle(z1,r);
path c2=circle(z2,r);
fill(c1,colour1);
fill(c2,colour2);

picture intersection=new picture;
fill(intersection,c1,colour1+colour2);
clip(intersection,c2);

add(intersection);

draw(c1);
draw(c2);

//draw("$\A$",box,z1);              // Requires [inline] package option.
//draw(Label("$\B$","$B$"),box,z2); // Requires [inline] package option.
draw("$A$",box,z1);            
draw("$\V{B}$",box,z2);

pair z=(0,-2);
real m=3;
margin BigMargin=Margin(0,m*dot(unit(z1-z),unit(z0-z)));

draw(Label("$A\cap B$",0),conj(z)--z0,Arrow,BigMargin);
draw(Label("$A\cup B$",0),z--z0,Arrow,BigMargin);
draw(z--z1,Arrow,Margin(0,m));
draw(z--z2,Arrow,Margin(0,m));

shipout(bbox(0.25cm));
\end{asy}
%\caption{Venn diagram}\label{venn}
\end{center}
%\end{figure}



























\Section{Method}
  Write introduction.

% Three parts
%   1. terminology (sonkal.taekwondo.cz)
%   2. english description (condensed encyclopedia)
%   3. tagging (how is tagging used? what is its role?)



\SubSection{The Data}

  Our data comes from two different sources. The terminology associated with
  movements are from Taekwon-Do ITF Sonkal
  Praha\footnote{\url{http://sonkal.taekwondo.cz}}, and the English
  description of each movement is from the condensed
  encyclopedia\cite{cyclo:con}.


\SubSection{Approach}
% collecting data

  Todo.


% registered terminology to all pattern movements, grouped by pattern names, including their pattern movement number, and motion type (normal, slow, etc.), link to movements_all.txt ?
%   reviewed the terminology and movement type with odd-magne using a wiki
%   made a few changes (corrected terminology for punch for instance)
% added english description from the condensed encyclopedia to pattern
%   checked that the terminology corresponded to the actual pattern movement
% added tags
%   (documented the tagging process)
%   give an overview over (groups of) tags, see movements_with_tags.txt
% divided each pattern into its own text file
% modelled the project in the database (./models)
%   explain all attributes etc.
% made a ruby script to parse the text files (link to bootstrap.rb ?)
%   adding the contents into a database using datamapper (orm) (link to datamapper) || populated the database using Datamapper (orm)
%   saving the data into an sqlite3 database for analysis
%
% show an example (chon-ji movement no. 1, for instance)
%
%
%
% The terminology to all pattern movements where first inserted into a text
% file, along with.. || including..
%

%   Tagging
%     Group of tags
%     Groups of techniques {blocks, strikes, thrusts, punches}
%       height
%       weapon
%       type etc.



% hva forholder jeg meg til ved utvetydigheter (15-binds vs. *kondenserte*)
% vurderinger på teknikker
% hvorfor er tekniker gruppert slik de er?











\Section{Results}

%What's the answer? Specifically, most good computer architecture papers
%conclude that something is so many percent faster, cheaper, smaller, or
%otherwise better than something else. Put the result there, in numbers.
%Avoid vague, hand-waving results such as ``very'', ``small'', or
%``significant.''

%If you must be vague, you are only given license to do so when you can
%talk about orders-of-magnitude improvement. There is a tension here in
%that you should not provide numbers that can be easily misinterpreted,
%but on the other hand you don't have room for all the caveats.

% Observation: Kicks happen between stances

\Section{Conclusions}

%What are the implications of your answer? Is it going to change the
%world (unlikely), be a significant ``win'', be a nice hack, or simply
%serve as a road sign indicating that this path is a waste of time (all
%of the previous results are useful). Are your results general,
%potentially generalizable, or specific to a particular case?


% Unnumbered section (note the '*')
\section*{Acknowledgments}

  % Thanks to Master Nicolaisen for pointing out that I allredy had found a
  % problem statement when I was still looking for one
  %
  % Odd-Magne Hansen for helping me with the terminology, and general feedback
  Todo.




\begin{table}
  \centering
  \begin{tabular}{l|c|c} \hline \hline
    & Left Arm & Right Arm \\ \hline

    Blocks    & 148   & 133 \\
    Punches   &  60   &  71 \\
    Strikes   &  34   &  38 \\
    Thrusts   &  21   &  14 \\
    \hline

    \textbf{Total}   & \textbf{263}   & \textbf{256} \\
    \hline
  \end{tabular}
  \caption{Total number of techniques per arm by category.}
  %\label{tab:table1}
  \end{table}


\begin{table}
  \centering
  \begin{tabular}{l|c|c|c} \hline \hline
    & High & Middle & Low \\ \hline

    Blocks    & x   & y   & z \\
    Punches   & x   & y   & z \\
    Strikes   & x   & y   & z \\
    Thrusts   & x   & y   & z \\
    \hline

  \end{tabular}
  \caption{Total number of techniques by height.}
  %\label{tab:table1}
  \end{table}


\begin{table}
  \centering
  \begin{tabular}{l|c|c|c} \hline \hline
    & High & Middle & Low \\ \hline

    Blocks    & x   & y   & z \\
    Punches   & x   & y   & z \\
    Strikes   & x   & y   & z \\
    Thrusts   & x   & y   & z \\
    \hline

  \end{tabular}
  \caption{Total number of techniques by height and arm.}
  %\label{tab:table1}
  \end{table}


% percent more useful than numbers?





% References
\begin{thebibliography}{99}
    \small  % Use 9 point text

    \bibitem{cyclo:vol1}
      Gen. Choi Hong Hi,
      \emph{Encyclopedia of Taekwon-Do}, Vol. 1,
      International Taekwon-Do Federation, 1993.

    \bibitem{cyclo:con}
      Gen. Choi Hong Hi,
      \emph{Taekwon-Do (the Korean Art of Self-Defence)}, Fifth Edition,
      International Taekwon-Do Federation, 1999.

    \bibitem{rlb}
      M. Gibb,
      \emph{Right Leg Bias in Patterns}, 2009.
      \url{http://visiontkd.co.uk/assetshome/pdf/Rightlegbias.pdf}

    \bibitem{golder:ct}
      Scott A. Golder, Bernardo A. Huberman,
      \emph{The Structure of Collaborative Tagging Systems}, 2005.
      \url{http://www.hpl.hp.com/research/scl/papers/tags/tags.pdf}

    \bibitem{heymann:ccch}
      Paul Heymann, Hector Garcia-Molina,
      \emph{Collaborative Creation of Communal Hierarchical Taxonomies in
      Social Tagging Systems}, 2006.
      \url{http://ilpubs.stanford.edu:8090/775/1/2006-10.pdf}

    \bibitem{huang:tt}
      Jeff Huang, Katherine M. Thornton, Efthimis N. Efthimiadis,
      ``Conversational Tagging in Twitter,''
      \emph{Hypertext}, pp. 173-177, 2010.

    \bibitem{jones:folders}
      William Jones, Ammy J. Phuwanartnurak, Rajdeep Gill, Harry Bruce,
      \emph{Don't Take My Folders Away! Organizing Personal Information to Get
      Things Done}, 2005.
      \url{https://dlib.lib.washington.edu/dspace/bitstream/handle/1773/2031/
           Don%27t%20take%20my%20folders%20away%2c%20current.pdf?sequence=2}

    \bibitem{mathes:folk}
      Adam Mathes,
      ``Folksonomies -- Cooperative Classification and Communication Through
      Shared Metadata,''
      \emph{Computer Mediated Communication, LIS590CMC}, 2004.

    \bibitem{shirky:ontology}
      Clay Shirky,
      \emph{Ontology is Overrated: Categories, Links, and Tags}, 2005.
      \url{http://www.shirky.com/writings/ontology_overrated.html}

    \bibitem{wikipedia:tags}
      \emph{Tag (metadata)}, July 12, 2010. \\
      \url{http://en.wikipedia.org/wiki/Tag}

\end{thebibliography}

\end{document}
