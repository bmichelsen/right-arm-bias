% Make two column format for LaTex 2e
\documentclass[10pt,twocolumn,a4paper]{article}

% Use the following packages
\usepackage{dcolumn}            % Align table columns on decimal point
\usepackage{graphicx}           % Include graphics
\usepackage{listings}           % Pretty source code listings
\usepackage{url}                % Handle URLs better

% Set dimensions of columns, gap between columns, and paragraph indent
\setlength{\textheight}{8.875in}
\setlength{\textwidth}{6.875in}
\setlength{\columnsep}{0.3125in}
\setlength{\topmargin}{0in}
\setlength{\headheight}{0in}
\setlength{\headsep}{0in}
\setlength{\parindent}{1pc}
\setlength{\oddsidemargin}{-.1875in}  % Centers text
\setlength{\evensidemargin}{-.1875in}

% Add the period after section numbers, and adjust spacing
\newcommand{\Section}[1]{\vspace{-8pt}\section{\hskip -1em.~~#1}\vspace{-3pt}}
\newcommand{\SubSection}[1]{\vspace{-3pt}\subsection{\hskip -1em.~~#1}
\vspace{-3pt}}

\begin{document}

% Make title bold
\title{\bf The Left Arm Bias in ITF Taekwon-Do Patterns \\ Chon-Ji to Juche}

% Single author
\author{Bjorn Michelsen, III. Dan \\
  \texttt{<bjorn@bmichelsen.no>} \\
  \\
  ITF Tromsdalen School of Taekwon-Do \\
  Tromso, Norway
  }

% Print date
\date{\today}

% Produce the title
\maketitle


\section*{\centering Abstract}
\begin{em}
  Patterns in ITF Taekwon-Do are defined as a choreographed sequence of
  fundamental movements, which represents an attack or defense towards a
  particular target area or definite action of an attacker, against one or
  several imaginary opponents.

  These patterns are said to be using every available attacking and
  blocking tool from different directions\cite{cyclo:vol1}. It has,
  however, recently been observered that a right leg bias exist\cite{rlb}.

  A study of arm techniques in patterns Chon-Ji to Juche, initiated in the
  hope of uncovering whether or not there is a bias to either arm, is
  reported here.
\end{em}

% Approach, Results, Conclusion.



% Introduction
%   - Problem Statement
% Method
%   - Approach
% Results
%   - Tables
%   - Graphs
% Conclusions


%Problem definition
%Is there a bias towards the right or the left arm in ITF Taekwon-Do
%patterns Chon-Ji to Juche?


\Section{Introduction}

% mer utdypende fra 'abstract', ta med ting fra Right Leg Bias om
% nødvendig osv. sette dette under Previous Work?

Overall, there are a total of 152 kicks. 85 kicks are with the right
leg, and 67 kicks are with the left leg. This means that 56\% of all
kicks in patterns are with the right leg, whereas 44\% are with the left
leg.

% note that Gibb's article includes *all* the patterns


\SubSection{Previous Work}

There are various bibliographic and citation schemes available in
LaTex, but we choose to use the simplest one in this example.

\Section{Method}

How did you go about solving or making progress on the problem? Did you
use simulation, analytic models, prototype construction, or analysis of
field data for an actual product? What was the extent of your work (did
you look at one application program or a hundred programs in twenty
different programming languages?) What important variables did you
control, ignore, or measure?


\Section{Results}

What's the answer? Specifically, most good computer architecture papers
conclude that something is so many percent faster, cheaper, smaller, or
otherwise better than something else. Put the result there, in numbers.
Avoid vague, hand-waving results such as ``very'', ``small'', or
``significant.''

If you must be vague, you are only given license to do so when you can
talk about orders-of-magnitude improvement. There is a tension here in
that you should not provide numbers that can be easily misinterpreted,
but on the other hand you don't have room for all the caveats.

\Section{Conclusions}

What are the implications of your answer? Is it going to change the
world (unlikely), be a significant ``win'', be a nice hack, or simply
serve as a road sign indicating that this path is a waste of time (all
of the previous results are useful). Are your results general,
potentially generalizable, or specific to a particular case?


% Unnumbered section (note the asteriks)
\section*{Acknowledgments}

Thank you, thank you, thank you!




\begin{table}
  \centering
  \begin{tabular}{l|c|c} \hline \hline
    & Left Arm & Right Arm \\ \hline

    Blocks    & 148   & 133 \\
    Punches   &  60   &  71 \\
    Strikes   &  34   &  38 \\
    Thrusts   &  21   &  14 \\
    \hline

    \textbf{Total}   & \textbf{263}   & \textbf{256} \\
    \hline
  \end{tabular}
  \caption{Total number of techniques per arm by category.}
  %\label{tab:table1}
  \end{table}


\begin{table}
  \centering
  \begin{tabular}{l|c|c|c} \hline \hline
    & High & Middle & Low \\ \hline

    Blocks    & x   & y   & z \\
    Punches   & x   & y   & z \\
    Strikes   & x   & y   & z \\
    Thrusts   & x   & y   & z \\
    \hline

  \end{tabular}
  \caption{Total number of techniques by height.}
  %\label{tab:table1}
  \end{table}


\begin{table}
  \centering
  \begin{tabular}{l|c|c|c} \hline \hline
    & High & Middle & Low \\ \hline

    Blocks    & x   & y   & z \\
    Punches   & x   & y   & z \\
    Strikes   & x   & y   & z \\
    Thrusts   & x   & y   & z \\
    \hline

  \end{tabular}
  \caption{Total number of techniques by height and arm.}
  %\label{tab:table1}
  \end{table}


% percent more useful than numbers?





% References
\begin{thebibliography}{99}
    \small  % Use 9 point text

    \bibitem{cyclo:vol1}
      General Choi Hong Hi,
      \emph{Encyclopedia of Taekwon-Do}, Vol. 1, p. 13,
      International Taekwon-Do Federation, 1983.

    \bibitem{rlb}
      M. Gibb,
      \emph{Right Leg Bias in Patterns}, 2009.
      \url{http://visiontkd.co.uk/assetshome/pdf/Rightlegbias.pdf}

%\bibitem{key:foo}
%I. M. Author,
%``Some Related Article I Wrote,''
%{\em Some Fine Journal}, Vol. 17, pp. 1-100, 1987.

%\bibitem{foo:baz}
%A. N. Expert,
%{\em A Book He Wrote,}
%His Publisher, 1989.











\end{thebibliography}

\end{document}
