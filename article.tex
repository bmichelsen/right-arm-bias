% Introduction
%   Definitions
%     Patterns
%     Fundamental movements
%  Related work
%  Problem statement
%     Groups of techniques
%       blocks
%       strikes
%       thrusts
%       punches
%     Tagging








  
% Method
%   Approach
%   Collecting data
%   Analysis of data
%   Tagging
%     Group of tags || Kind of Tags
%       height
%       weapon
%       type etc.
%    Difficulties related to tagging
%  Uncertainty
%
% Results
%   Tables
%   Graphs
% Conclusion


%Related work
%  Right Leg Bias













% Make two column format for LaTex 2e
\documentclass[10pt,twocolumn,a4paper]{article}

% Use the following packages
\usepackage[utf8]{inputenc}     % Support for UTF-8
\usepackage{dcolumn}            % Align table columns on decimal point
\usepackage{graphicx}           % Include graphics
\usepackage{listings}           % Pretty source code listings
\usepackage{url}                % Handle URLs better

% Set dimensions of columns, gap between columns, and paragraph indent
\setlength{\textheight}{8.875in}
\setlength{\textwidth}{6.875in}
\setlength{\columnsep}{0.3125in}
\setlength{\topmargin}{0in}
\setlength{\headheight}{0in}
\setlength{\headsep}{0in}
\setlength{\parindent}{1pc}
\setlength{\oddsidemargin}{-.1875in}  % Centers text
\setlength{\evensidemargin}{-.1875in}

% Add the period after section numbers, and adjust spacing
\newcommand{\Section}[1]{\vspace{-8pt}\section{\hskip -1em.~~#1}\vspace{-3pt}}
\newcommand{\SubSection}[1]{\vspace{-3pt}\subsection{\hskip -1em.~~#1}
\vspace{-3pt}}

\begin{document}

% Make the title bold
\title{\bf The Left Arm Bias in ITF Taekwon-Do Patterns \\ Chon-Ji to Juche}

% Author information
\author{Bjørn Michelsen, III. Dan \\
  \texttt{<bjorn@bmichelsen.no>} \\
  \\
  ITF Tromsdalen School of Taekwon-Do \\
  Tromsø, Norway
  }

% Print date
\date{\today}

% Produce the title
\maketitle


\section*{\centering Abstract}
\begin{em}

%An academic abstract typically outlines four elements germane to the completed work:

    %* The research focus (i.e. statement of the problem(s)/research issue(s) addressed);
    %* The research methods used (experimental research, case studies, questionnaires, etc.);
    %* The results/findings of the research; and
    %* The main conclusions and recommendations

  % see the article: The Structure of Collaborative Tagging Systems
  % located in ~/tmp/tags.pdf

  Todo.
\end{em}





\Section{Introduction}

  A \emph{pattern} in International Taekwon-Do Federation (ITF) is defined as
  a choreographed sequence of fundamental movements. These \emph{fundamental
  movements} are basic elements which represent an attack or defense towards a
  specific target area, or a predetermined action of an attacker.

  Combined in a pattern, it allows the student to systematically deal with one
  or several imaginary opponents under various assumptions, using every
  available attacking and blocking tool from different directions. In
  addition, the techniques should also be evenly distributed between the left
  and the right side\cite{cyclo:vol1}.

  It has, however, recently been observed that a bias towards the right leg
  exist\cite{rlb}. % ALL patterns..

% how is related work written about?


% recent studies of a right leg bias; Recent work has found that..
%
% in this paper, therefore, we analyze..

%It has, however, recently been observed that a right leg bias
%exist\cite{rlb} when looking at all the 24 Taekwon-Do patterns.

% Is there a bias towards the *right* leg in Chon-Ji to Juche?

%A study of arm techniques in patterns Chon-Ji to Juche, initiated in the hope
%of uncovering whether or not there is a bias to either arm, is reported here.



% Problem Statement
%
% pp. 154-155
%
% The following points should be considered while performing patterns:
%
%   9. Attack and defense techniques should be equally
%      distributed among right and left hands and feet
%
%
%
% and the recent discovery of the right leg bias,
%
% Is there a bias towards the right or the left arm in ITF Taekwon-Do
% patterns Chon-Ji to Juche?
%
% note that Gibb's article includes *all* the patterns


%Overall, there are a total of 152 kicks. 85 kicks are with the right
%leg, and 67 kicks are with the left leg. This means that 56\% of all
%kicks in patterns are with the right leg, whereas 44\% are with the left
%leg.


\Section{Method}

How did you go about solving or making progress on the problem? Did you
use simulation, analytic models, prototype construction, or analysis of
field data for an actual product? What was the extent of your work (did
you look at one application program or a hundred programs in twenty
different programming languages?) What important variables did you
control, ignore, or measure?

% Tagging can be defined as the practice of creating and
% managing labels that categorize content using simple
% keywords.

\Section{Results}

What's the answer? Specifically, most good computer architecture papers
conclude that something is so many percent faster, cheaper, smaller, or
otherwise better than something else. Put the result there, in numbers.
Avoid vague, hand-waving results such as ``very'', ``small'', or
``significant.''

If you must be vague, you are only given license to do so when you can
talk about orders-of-magnitude improvement. There is a tension here in
that you should not provide numbers that can be easily misinterpreted,
but on the other hand you don't have room for all the caveats.

\Section{Conclusions}

What are the implications of your answer? Is it going to change the
world (unlikely), be a significant ``win'', be a nice hack, or simply
serve as a road sign indicating that this path is a waste of time (all
of the previous results are useful). Are your results general,
potentially generalizable, or specific to a particular case?


% Unnumbered section (note the '*')
\section*{Acknowledgments}

  % Odd-Magne Hansen for helping me with the terminology, and general feedback
  Todo.




\begin{table}
  \centering
  \begin{tabular}{l|c|c} \hline \hline
    & Left Arm & Right Arm \\ \hline

    Blocks    & 148   & 133 \\
    Punches   &  60   &  71 \\
    Strikes   &  34   &  38 \\
    Thrusts   &  21   &  14 \\
    \hline

    \textbf{Total}   & \textbf{263}   & \textbf{256} \\
    \hline
  \end{tabular}
  \caption{Total number of techniques per arm by category.}
  %\label{tab:table1}
  \end{table}


\begin{table}
  \centering
  \begin{tabular}{l|c|c|c} \hline \hline
    & High & Middle & Low \\ \hline

    Blocks    & x   & y   & z \\
    Punches   & x   & y   & z \\
    Strikes   & x   & y   & z \\
    Thrusts   & x   & y   & z \\
    \hline

  \end{tabular}
  \caption{Total number of techniques by height.}
  %\label{tab:table1}
  \end{table}


\begin{table}
  \centering
  \begin{tabular}{l|c|c|c} \hline \hline
    & High & Middle & Low \\ \hline

    Blocks    & x   & y   & z \\
    Punches   & x   & y   & z \\
    Strikes   & x   & y   & z \\
    Thrusts   & x   & y   & z \\
    \hline

  \end{tabular}
  \caption{Total number of techniques by height and arm.}
  %\label{tab:table1}
  \end{table}


% percent more useful than numbers?





% References
\begin{thebibliography}{99}
    \small  % Use 9 point text

    \bibitem{cyclo:vol1}
      Gen. Choi Hong Hi,
      \emph{Encyclopedia of Taekwon-Do}, Vol. 1,
      International Taekwon-Do Federation, 1993.

    \bibitem{rlb}
      M. Gibb,
      \emph{Right Leg Bias in Patterns}, 2009.
      \url{http://visiontkd.co.uk/assetshome/pdf/Rightlegbias.pdf}

\end{thebibliography}

\end{document}
