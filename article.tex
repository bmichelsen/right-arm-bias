% Make two column format for LaTex 2e
\documentclass[10pt,twocolumn,a4paper]{article}

% Use the following packages
\usepackage[utf8]{inputenc}     % Support for UTF-8
\usepackage{dcolumn}            % Align table columns on decimal point
\usepackage{graphicx}           % Include graphics
\usepackage{asymptote}          % Vector graphics language
\usepackage{listings}           % Pretty source code listings
\usepackage{url}                % Handle URLs better
\usepackage{amsmath}            % Extend mathematical typesetting

% Set dimensions of columns, gap between columns, and paragraph indent
\setlength{\textheight}{8.875in}
\setlength{\textwidth}{6.875in}
\setlength{\columnsep}{0.3125in}
\setlength{\topmargin}{0in}
\setlength{\headheight}{0in}
\setlength{\headsep}{0in}
\setlength{\parindent}{1pc}
\setlength{\oddsidemargin}{-.1875in}    % Centers the text
\setlength{\evensidemargin}{-.1875in}

% Add the period after section numbers, and adjust spacing
\newcommand{\Section}[1]{\vspace{-8pt}\section{\hskip -1em.~~#1}\vspace{-3pt}}
\newcommand{\SubSection}[1]{\vspace{-3pt}\subsection{\hskip -1em.~~#1}
\vspace{-3pt}}

% Global Asymptote definitions
\begin{asydef}
import three;
usepackage("bm");
texpreamble("\def\V#1{\bm{#1}}");
\end{asydef}

% Document
\begin{document}

% Make the title bold
\title{\bf The Left Arm Bias in ITF Taekwon-Do Patterns \\ Chon-Ji to Juche}

% Author information
\author{Bjørn Michelsen, III. Dan \\
  \texttt{<bjorn@bmichelsen.no>} \\
  \\
  ITF Tromsdalen School of Taekwon-Do \\
  Tromsø, Norway
  }

% Print date
\date{\today}

% Produce the title
\maketitle

% Sections
\section*{\centering Abstract}
\begin{em}

%An academic abstract typically outlines four elements germane to the completed work:

    %* The research focus (i.e. statement of the problem(s)/research issue(s) addressed);
    %* The research methods used (experimental research, case studies, questionnaires, etc.);
    %* The results/findings of the research; and
    %* The main conclusions and recommendations

  Todo.
\end{em}


\Section{Introduction}

  A \emph{pattern} in International Taekwon-Do Federation (ITF) is defined as
  a choreographed sequence of fundamental movements. These \emph{fundamental
  movements} are basic elements which represent an attack or defense against a
  specific target area, or a predetermined action of an attacker.

  Combined in a pattern, it allows the student to systematically deal with one
  or several imaginary opponents under various assumptions, using every
  available attacking and blocking tool from different directions. In
  addition, the techniques should also be evenly distributed between the left
  and the right side\cite{cyclo:vol1}.

  It has, however, recently been observed that a bias towards the right leg
  exist. The study by Gibbs\cite{rlb} show that there is a total of 152 kicks
  in patterns Chon-Ji to Tong-Il. 85 kicks are with the right leg, and 67
  kicks are with the left leg. This means that 56\% of the kicks are with the
  right leg, and 44\% with the left leg.

  The work by Gibbs focused on kicks in all patterns, whereas this article
  examines patterns Chon-Ji to Juche. Looking at kicks in said patterns using
  Gibbs' paper, we find that there are 72 kicks in total. 42 kicks are with
  the right leg, and 30 kicks are with the left leg. Roughly 58\% of the kicks
  are with the right leg, and about 42\% with the left. Thus, the right leg
  bias still holds.

  In this article we analyze hand techniques in patterns Chon-Ji to Juche
  using tags, and then report our findings of a left arm bias.


\SubSection{Tags}

  In a hierarchical and exclusive classification system, each object belong in
  one unambiguous category, which in turn is within another more general one.

  Take the hierarchy of folders in a computer file system, for example. How we
  choose to organize the folders reflects a decision concerning the relative
  importance of each characteristic\cite{golder:ct}.

  Folder names are in themselves informative, in that, like tags, they
  describe the information held within them\cite{jones:folders}. Furthermore,
  they way folder names relate to each other show a structural relationship
  between directories\cite{mathes:folk}.

  \emph{Tags}, on the other hand, are non-hierarchical descriptive terms,
  keywords or labels that is attached to an object for later retrieval
  \cite{golder:ct}\cite{huang:tt}\cite{shirky:ontology}.

  They are \emph{non-hierarchical} and \emph{inclusive}, which is to say that
  an object can be associated with a great variety of tags
  simultaneously\cite{golder:ct}. The \emph{descriptive terms},
  \emph{keywords} or \emph{labels} are metadata connected to a given object
  that describes a concept or type of information\cite{heymann:ccch}.


\SubSection{Tagging}

  \emph{Tagging} is the act of organizing a collection of objects into related
  groups\cite{shirky:ontology}.


\SubSection{Set theory}
  A \emph{set} is a collection of distinct objects. It is ``a plurality
  thought of as a unit''\cite{hausdorff:sets}. \emph{Set theory}, then, is the
  branch of mathematics that studies sets.

  An object which has been tagged contains a collection, or a set, of tags.
  Each tag is an element of the set. The order of elements within the set is
  irrelevant, but each element can only occur once.

  For example, $\{1,5,8\}$ is a set with three elements: 1, 5 and 8.
  $\{5,8,1\}$ is the same set, since the order of the elements doesn't matter.
  $\{1,1,1\}$, however, is not a valid set because the element 1 occurs
  repeatedly. Elements in a set can contain strings as well, so $\{blocks,
  punches, strikes, thrusts\}$ is also a valid set.

  We usually denote a set by a capital letter for reference purposes. For
  instance, $S=\{1,5,8\}$.

  $a \in S$ denotes that the object $a$ is a member of the set $S$. If $a$
  is not a member of S, we write $a \not \in S$.

  The \emph{union} of two sets $S$ and $T$ is the set

  \[
    S \cup T = \{x:x \in S \: or \: x \in T\}
  \]

  % Venn diagram, showing the union of two sets
  \input{venn_union.asy}

  This consists of those objects which lie in the set $S$ or in the set $T$,
  or in both. So,

  \[
    \{1, 2, 3\} \cup \{3, 4, 5\} = \{1, 2, 3, 4, 5\}
  \]

  The \emph{intersection} of two sets $S$ and $T$ is the set

  \[
    S \cap T = \{x:x \in S \: and \: x \in T\}
  \]

  % Venn diagram, showing the intersect of two sets
  \input{venn_intersect.asy}

  This consists of those objects which lie in both the set $S$ and the set
  $B$. So,

  \[
    \{1, 2, 3\} \cap \{2, 3, 4\} = \{2, 3\}
  \]

  The \emph{complement} of a set $T$ relative to a set $S$ is the set

  \[
    S - T = \{x:x \in S \: and \: x \not \in T\}
  \]

  % Venn diagram, showing the complement of two sets
  \input{venn_complement.asy}

  This consists of those objects which lie in the set $S$ but not in the set
  $T$. So,

  \[
    \{1, 2, 3\} - \{2, 3, 4\} = \{1\}
  \]

  The \emph{symmetric difference} of the sets $S$ and $T$ is the set

  \[
    S \Delta T = \{x:(x \in S \: and \: x \not \in T) \: or \: (x \in T \: and
    \: x \not \in S)\}
  \]

  % Venn diagram, showing the symmetric difference of two sets
  \input{venn_difference.asy}

  This consists of those objects which lie either in one of the sets, but not
  in both. So,

  \[
    \{1, 2, 3\} \Delta \{2, 3, 4\} = \{1, 4\}
  \]


\Section{Method}

  In this section we describe where the data originates. Next we elaborate on
  the approach taken to register and investigate the dataset. Finally, we have
  a look at how each movement has been classified with emphasis on special
  cases.


\SubSection{The Data}

  Our data comes from two different sources. The terminology associated with
  pattern movements is from Taekwon-Do ITF Sonkal Praha\footnote{\url{http://
  sonkal.taekwondo.cz}}, and the English description of each movement is from
  the condensed encyclopedia\cite{cyclo:con}.

  The dataset is organized by pattern names. It contains a list of all
  techniques in patterns Chon-Ji to Juche. For each movement, the following
  data has been registered:

  \begin{enumerate}
    \item Movement number
    \item English description
    \item Terminology
    \item Tags
  \end{enumerate}


\SubSection{Approach}

  The movement number, together with the English description of each movement,
  was registered manually from the condensed encyclopedia. After reviewing and
  making minor corrections to the terminology, it too was added to the
  dataset.
  
  Suitable tags were attached to every movement by first extracting keywords
  from their English description, and then adding more detailed tags while
  performing the pattern.

  We then defined the data models with DataMapper, an Object Relational
  Mapper\footnote{\url{http://www.datamapper.org}} (ORM) written in Ruby. The
  models are Ruby classes with properties and associations. Properties define
  field names and data types in the database, whereas the associations defines
  the relationships and cardinality between the models.

  Next we created a Ruby script that reads the dataset and automatically
  populates the database. It sets up associations between models, and adds the
  appropriate tags to every movement, as well as associations between movement
  and pattern.

  Tagging movements in such a formalized manner, and registering it in a
  database, enables us to retrieve the data we are interested in using
  concepts from set theory.


\SubSection{Classification}

  Classification is done by attaching metadata, in terms of tags, to each
  pattern movement.

  The first movement in Chon-Ji, for instance, is to ``[m]ove the left foot to
  B, forming a left walking stance toward B while executing a low block to B
  with the left forearm\cite{cyclo:vol1}.'' Removing the verbose parts of the
  description leaves us with:

  \[
    \{left \: walking \: stance, \: low \: block, \: left \: forearm\}
  \]

  Supplementing the above with more detailed tags gives us:

  \begin{align*}
    \{walking \: stance, \: left \: stance, \: low \: technique, \\
    left \: technique, \: block, \: forearm, \\
    outer \: forearm, normal \: motion\}
  \end{align*}

  Information about the stance type is presented first, followed by stance
  side (left, right or none where applicable --- for example a sitting
  stance), technique height (low, middle, high), technique side (left arm or
  right arm, it can also be both), direction of the technique (when needed),
  primary technique type (attack, block, kick), technique group (punch,
  strike, thrust), tool group (forearm, knife hand, finger etc.), tool (inner
  forearm, outer forearm, double finger etc.), and finally motion type (normal
  motion, slow motion, consecutive motion and so on).

  In most cases, applying tags to movements have been straightforward.
  However, there are movements that need special mention.  These movements
  fall into two categories.

  The first category consist of movements where both arms are part of the
  technique by either forming the technique, or by being two separate
  techniques in a single movement.

  When both arms are used in one technique, the movement has been tagged with
  $\{left\_technique, \: right\_technique\}$ to reflect that both arms are
  needed for the technique, or that the movement contains two separate hand
  techniques.

  An example of a movement where both arms are required to form the technique
  can be found in the 13\textsuperscript{th} movement of Joong-Gun Tul, which
  is \emph{Gunnun so kyocha joomuk chukyo makgi}.

  When it comes to a movement with two separate hand techniques, the
  27\textsuperscript{th} movement of Joong-Gun Tul, being \emph{Nachuo so
  sonbadak noollo makgi}, serves as a nice illustration.

  The implications of tagging movements with $\{left\_technique, \:
  right\_technique\}$ is that the total amount of hand techniques on either
  side are higher than what actually is the case. This flaw does not affect
  the bias we are investigating.

  The second category comprises ambiguous movements, and thus contain
  clarification on why certain tags were added.

  Below is a list of all the special cases from both categories.

\begin{itemize}
  \item
    \emph{Do-San Tul}
    \begin{itemize}
      \item
        {\bf Movement 13}, \emph{Gunnun so bakat palmok nopunde hechyo makgi},
        is tagged as a left and a right technique with $\{left\_technique, \:
        right\_technique\}$. This increases the total amount of blocks on both
        sides.
       \item
        {\bf Movement 17}, \emph{Gunnun so bakat palmok nopunde hechyo makgi},
        is tagged as a left and a right technique with $\{left\_technique, \:
        right\_technique\}$. This increases the total amount of blocks on both
        sides.
    \end{itemize}
  \item
    \emph{Yul-Gok Tul}
    \begin{itemize}
      \item
        {\bf Movement 24}, \emph{Gunnun so ap palkup bandae taerigi}, is
        tagged with $\{front\_strike\}$ due to the English description of
        striking with the right front elbow.
      \item
        {\bf Movement 27}, \emph{Gunnun so ap palkup bandae taerigi}, is
        tagged with $\{front\_strike\}$ due to the English description of
        striking with the left front elbow.
    \end{itemize}
  \item
    \emph{Joong-Gun Tul}
    \begin{itemize}
      \item
        {\bf Movement 8}, \emph{Gunnun so wipalkup taerigi}, is tagged
        $\{upper\_strike, \: upper\_elbow\}$ since the English description
        says that the movement is ``a right upper elbow strike.''
      \item
        {\bf Movement 10}, \emph{Gunnun so wipalkup taerigi}, is tagged
        $\{upper\_strike, \: upper\_elbow\}$ since the English description
        says that the movement is ``a left upper elbow strike.''
      \item
        {\bf Movement 11}, \emph{Gunnun so sang sewo jirugi}, is tagged as a
        left and right technique, with $\{left\_technique, \:
        right\_technique\}$. This increases the total amount of punches on
        both sides.
      \item
        {\bf Movement 12}, \emph{Gunnun so sang dwijibo jirugi}, is tagged as
        a left and right technique, with $\{left\_technique, \:
        right\_technique\}$. This increases the total amount of punches on
        both sides.
      \item
        {\bf Movement 13}, \emph{Gunnun so kyocha joomuk chukyo makgi}, is
        tagged as a left and a right technique with $\{left\_technique, \:
        right\_technique\}$. This increases the total amount of blocks on both
        sides.
      \item
        {\bf Movement 27}, \emph{Nachuo so sonbadak noollo makgi}, is tagged as
        a $\{middle\_technique, \: low\_technique\}$ because the left hand is
        in the middle section and the right hand, forming the pressing block,
        is in the low section of the body. The movement has, additionally, been
        tagged as a $\{right\_technique, \: left\_technique\}$ to reflect that
        two separate blocks are executed.
      \item
        {\bf Movement 29}, \emph{Nachuo so sonbadak noollo makgi}, is tagged as
        a $\{middle\_technique, \: low\_technique\}$ because the right hand is
        in the middle section and the left hand, forming the pressing block, is
        in the low section of the body. The movement has, additionally, been
        tagged as a $\{right\_technique, \: left\_technique\}$ to reflect that
        two separate blocks are executed.
    \end{itemize}
  \item
    \emph{Toi-Gae Tul}
    \begin{itemize}
      \item
        {\bf Movement 7}, \emph{Gunnun so kyocha joomuk noollo makgi}, is
        tagged as a left and a right technique with $\{left\_technique, \:
        right\_technique\}$. This increases the total amount of blocks on both
        sides.
      \item
        {\bf Movement 8}, \emph{Gunnun so sang joomuk nopunde sewo jirugi}, is
        tagged as a left and right technique, with $\{left\_technique, \:
        right\_technique\}$. This increases the total amount of punches on
        both sides.
      \item
        {\bf Movement 20}, \emph{Gunnun sogi}, has only the stance registered
        as part of the terminology though the English description says to
        ``[e]xtend both hands upward as if to grab the opponent's head.''
      \item
        {\bf Movement 29}, \emph{Kyocha so kyocha joomuk noollo makgi}, is
        tagged as a left and a right technique with $\{left\_technique, \:
        right\_technique\}$. This increases the total amount of blocks on both
        sides.
    \end{itemize}
  \item
    \emph{Hwa-Rang Tul}
    \begin{itemize}
      \item
        {\bf Movement 24}, \emph{Gunnun so kyocha joomuk noollo makgi}, is
        tagged as a left and a right technique with $\{left\_technique, \:
        right\_technique\}$. This increases the total amount of blocks on both
        sides.
    \end{itemize}
  \item
    \emph{Choong-Moo Tul}
    \begin{itemize}
      \item
        {\bf Movement 11}, \emph{Gunnun sogi}, has only the stance registered
        as part of the terminology though the English description says to
        ``[e]xtend both hands upward as if to grab the opponent's head.''
      \item
        {\bf Movement 27}, \emph{Niunja so kyocha sonkal kaunde momcho makgi},
        is tagged as a left and a right technique with $\{left\_technique, \:
        right\_technique\}$. This increases the total amount of blocks on both
        sides.
    \end{itemize}
  \item
    \emph{Kwang-Gae Tul}
    \begin{itemize}
      \item
        {\bf Movement 21}, \emph{Nachuo so sonbadak noollo makgi}, is tagged as
        a $\{middle\_technique, \: low\_technique\}$ because the right hand is
        in the middle section and the left hand, forming the pressing block, is
        in the low section of the body. The movement has, additionally, been
        tagged as a $\{right\_technique, \: left\_technique\}$ to reflect that
        two separate blocks are executed.
      \item
        {\bf Movement 22}, \emph{Nachuo so sonbadak noollo makgi}, is tagged as
        a $\{middle\_technique, \: low\_technique\}$ because the right hand is
        in the middle section and the left hand, forming the pressing block, is
        in the low section of the body. The movement has, additionally, been
        tagged as a $\{right\_technique, \: left\_technique\}$ to reflect that
        two separate blocks are executed.
      \item
        {\bf Movement 31}, \emph{Gunnun so sang sewo jirugi}, is tagged as a
        left and right technique, with $\{left\_technique, \:
        right\_technique\}$. This increases the total amount of punches on
        both sides.
      \item
        {\bf Movement 32}, \emph{Gunnun so sang joomuk dwijibo jirugi}, is
        tagged as a left and right technique, with $\{left\_technique, \:
        right\_technique\}$. This increases the total amount of punches on
        both sides.
      \item
        {\bf Movement 36}, \emph{Gunnun so sang joomuk dwijibo jirugi}, is
        tagged as a left and right technique, with $\{left\_technique, \:
        right\_technique\}$. This increases the total amount of punches on
        both sides.
    \end{itemize}
  \item
    \emph{Po-Eun Tul}
    \begin{itemize}
      \item
        {\bf Movement 6}, \emph{Annun so ap joomuk noollo makgi}, consist of
        two blocks. The first is a ``pressing block with the left fore fist,''
        and the second one is a ``side front block with the right inner
        forearm.'' When searching, this movement will only register as one
        block.
      \item
        {\bf Movement 7}, \emph{Annun so ap joomuk noollo makgi}, consist of
        two blocks. The first is a ``pressing block with the right fore
        fist,'' and the second one is a ``side front block with the left inner
        forearm.'' When searching, this movement will only register as one
        block.
      \item
        {\bf Movement 8}, \emph{Annun so an palmok kaunde hechyo makgi}, is
        tagged as a left and a right technique with $\{left\_technique, \:
        right\_technique\}$. This increases the total amount of blocks on both
        sides.
      \item
        {\bf Movement 24}, \emph{Annun so ap joomuk noollo makgi}, consist of
        two blocks. The first is a ``pressing block with the right fore
        fist,'' and the second one is a ``side front block with the left inner
        forearm.'' When searching, this movement will only register as one
        block.
      \item
        {\bf Movement 25}, \emph{Annun so ap joomuk noollo makgi}, consist of
        two blocks. The first is a ``pressing block with the left fore fist,''
        and the second one is a ``side front block with the right inner
        forearm.'' When searching, this movement will only register as one
        block.
      \item
        {\bf Movement 26}, \emph{Annun so an palmok kaunde hechyo makgi}, is
        tagged as a left and a right technique with $\{left\_technique, \:
        right\_technique\}$. This increases the total amount of blocks on both
        sides.
    \end{itemize}

    Movements 6, 7, 24 and 25 for pattern Po-Eun also have the following tags
    attached:

  \begin{align*}
    \{sitting\_stance, \: low\_technique, \: high\_technique, \\
    left\_technique, \: right\_technique, \: block, \\
    pressing\_block, fore\_fist, \: forearm, \\
    inner\_forearm, \: side\_front\_block, \\
    continuous\_motion\}
  \end{align*}

  \item
    \emph{Ge-Baek Tul}
    \begin{itemize}
      \item
        {\bf Movement 1}, \emph{Niunja so kyocha sonkal kaunde momcho makgi},
        is tagged as a left and a right technique with $\{left\_technique, \:
        right\_technique\}$. This increases the total amount of blocks on both
        sides.
      \item
        {\bf Movement 7}, \emph{Gunnun so nopunde doo bandalson makgi}, is
        tagged as a left and a right technique with $\{left\_technique, \:
        right\_technique\}$. This increases the total amount of blocks on both
        sides.
      \item
        {\bf Movement 20}, \emph{Annun so gutja makgi}, is tagged as a left
        and a right technique with $\{left\_technique, \: right\_technique\}$.
        This increases the total amount of blocks on both sides.
      \item
        {\bf Movement 23}, \emph{Twimyo yopcha jirugi}, the height of the kick
        is not registered.
      \item
        {\bf Movement 24}, \emph{Gunnun so sang sewo jirugi}, is tagged as a
        left and right technique, with $\{left\_technique, \:
        right\_technique\}$. This increases the total amount of punches on
        both sides.
      \item
        {\bf Movement 25}, \emph{Gunnun so nopunde doo bandalson makgi}, is
        tagged as a left and a right technique with $\{left\_technique, \:
        right\_technique\}$. This increases the total amount of blocks on both
        sides.
      \item
        {\bf Movement 34}, \emph{Gunnun so sang sewo jirugi}, is tagged as a
        left and right technique, with $\{left\_technique, \:
        right\_technique\}$. This increases the total amount of punches on
        both sides.
      \item
        {\bf Movement 36}, \emph{Annun so gutja makgi}, is tagged as a left
        and a right technique with $\{left\_technique, \: right\_technique\}$.
        This increases the total amount of blocks on both sides.
    \end{itemize}
  \item
    \emph{Eui-Am Tul}
    \begin{itemize}
      \item
        {\bf Movement 5}, \emph{Gunnun so kyocha joomuk naeryo makgi}, is
        tagged as a left and a right technique with $\{left\_technique, \:
        right\_technique\}$. This increases the total amount of blocks on both
        sides.
      \item
        {\bf Movement 18}, \emph{Gunnun so kyocha joomuk naeryo makgi}, is
        tagged as a left and a right technique with $\{left\_technique, \:
        right\_technique\}$. This increases the total amount of blocks on both
        sides.
      \item
        {\bf Movement 27}, \emph{Gunnun so sonkal kaunde hechyo makgi}, is
        tagged as a left and a right technique with $\{left\_technique, \:
        right\_technique\}$. This increases the total amount of blocks on both
        sides.
      \item
        {\bf Movement 29}, \emph{Dwitbal so sang sonbadak naeryo makgi}, is
        tagged as a left and a right technique with $\{left\_technique, \:
        right\_technique\}$. This increases the total amount of blocks on both
        sides.
      \item
        {\bf Movement 32}, \emph{Gunnun so sonkal kaunde hechyo makgi}, is
        tagged as a left and a right technique with $\{left\_technique, \:
        right\_technique\}$. This increases the total amount of blocks on both
        sides.
      \item
        {\bf Movement 34}, \emph{Dwitbal so sang sonbadak naeryo makgi}, is
        tagged as a left and a right technique with $\{left\_technique, \:
        right\_technique\}$. This increases the total amount of blocks on both
        sides.
    \end{itemize}
  \item
    \emph{Choong-Jang Tul}
    \begin{itemize}
      \item
        {\bf Movement 18}, \emph{Gunnun so kyocha joomuk noollo makgi}, is
        tagged as a left and a right technique with $\{left\_technique, \:
        right\_technique\}$. This increases the total amount of blocks on both
        sides.
      \item
        {\bf Movement 19}, \emph{Moorup najunde apcha busigi}, is tagged with
        $\{front\_snap\_kick, \: knee\}$ because the English description says
        to perform ``a low front snap kick to C with the right knee.'' This
        has an effect on the total amount of low front snap kicks.
      \item
        {\bf Movement 24}, \emph{Dwitbal so sang sonbadak noollo makgi}, is
        tagged as a left and a right technique with $\{left\_technique, \:
        right\_technique\}$. This increases the total amount of blocks on both
        sides.
      \item
        {\bf Movement 38}, \emph{Gunnun so bandae gutja makgi}, is tagged as a
        left and a right technique with $\{left\_technique, \:
        right\_technique\}$.  This increases the total amount of blocks on
        both sides.
      \item
        {\bf Movement 40}, \emph{Gunnun so bandae gutja makgi}, is tagged as a
        left and a right technique with $\{left\_technique, \:
        right\_technique\}$.  This increases the total amount of blocks on
        both sides.
      \item
        {\bf Movement 41}, \emph{Gunnun so sang sonkal soopyong taerigi}, is
        tagged with $\{knife\_hand, \: twin\_knife\_hand\}$.
    \end{itemize}
  \item
    \emph{Juche Tul}
    \begin{itemize}
      \item
        {\bf Movement 1}, \emph{Annun so an palmok narani makgi}, is tagged as
        left and a right technique with $\{left\_technique, \:
        right\_technique\}$. This increases the total amount of blocks on
        both sides.
      \item
        {\bf Movement 4}, \emph{Waebal so bakat palmok narani makgi}, is
        tagged as a left and a right technique with $\{left\_technique, :\
        right\_technique\}$. This increases the total amount of blocks on both
        sides. The movement also has the tag $\{parallel\_block\}$ attached to
        it.
      \item
        {\bf Movement 13}, \emph{Annun so an palmok narani makgi}, is tagged
        as a left and a right technique with $\{left\_technique, \:
        right\_technique\}$. This increases the total amount of blocks on both
        sides.
      \item
        {\bf Movement 16}, \emph{Waebal so bakat palmok narani makgi}, is
        tagged as a left and a right technique with $\{left\_technique, :\
        right\_technique\}$. This increases the total amount of blocks on both
        sides. The movement also has the tag $\{parallel\_block\}$ attached to
        it.
      \item
        {\bf Movement 38}, \emph{Sasun so sang sonbadak chookyo makgi}, is
        tagged as a left and a right technique with $\{left\_technique, \:
        right\_technique\}$. This increases the total amount of blocks on both
        sides.
    \end{itemize}
\end{itemize}


































%\Section{Results}

%What's the answer? Specifically, most good computer architecture papers
%conclude that something is so many percent faster, cheaper, smaller, or
%otherwise better than something else. Put the result there, in numbers.
%Avoid vague, hand-waving results such as ``very'', ``small'', or
%``significant.''

%If you must be vague, you are only given license to do so when you can
%talk about orders-of-magnitude improvement. There is a tension here in
%that you should not provide numbers that can be easily misinterpreted,
%but on the other hand you don't have room for all the caveats.

% Observation: Kicks happen between stances
% Tables
% Graphs

%\Section{Conclusions}

%What are the implications of your answer? Is it going to change the
%world (unlikely), be a significant ``win'', be a nice hack, or simply
%serve as a road sign indicating that this path is a waste of time (all
%of the previous results are useful). Are your results general,
%potentially generalizable, or specific to a particular case?


% Is it possible to say anything about why there are more blocks with the left
% hand, than with the right?


% Unnumbered section (note the '*')
%\section*{Acknowledgments}

  % Thanks to Master Nicolaisen (VII. Dan) for pointing out that I already had
  % found a problem statement when I was still looking for one
  %
  % Odd-Magne Hansen (III. Dan) for helping me with the terminology, and
  % general feedback
  %Todo.




\begin{table}
  \centering
  \begin{tabular}{l|c|c} \hline \hline
    & Left Arm & Right Arm \\ \hline

    Blocks    & 148   & 133 \\
    Punches   &  60   &  71 \\
    Strikes   &  34   &  38 \\
    Thrusts   &  21   &  14 \\
    \hline

    \textbf{Total}   & \textbf{263}   & \textbf{256} \\
    \hline
  \end{tabular}
  \caption{Total number of techniques per arm by category.}
  %\label{tab:table1}
  \end{table}







% References
\begin{thebibliography}{99}
    \small  % Use 9 point text

    \bibitem{cyclo:vol1}
      Gen. Choi Hong Hi,
      \emph{Encyclopedia of Taekwon-Do}, Vol. 1,
      International Taekwon-Do Federation, 1993.

    \bibitem{cyclo:con}
      Gen. Choi Hong Hi,
      \emph{Taekwon-Do (the Korean Art of Self-Defence)}, Fifth Edition,
      International Taekwon-Do Federation, 1999.

    \bibitem{rlb}
      M. Gibb,
      \emph{Right Leg Bias in Patterns}, 2009.
      \url{http://visiontkd.co.uk/assetshome/pdf/Rightlegbias.pdf}

    \bibitem{golder:ct}
      S. A. Golder, B. A. Huberman,
      \emph{The Structure of Collaborative Tagging Systems}, 2005.
      \url{http://www.hpl.hp.com/research/scl/papers/tags/tags.pdf}

    \bibitem{heymann:ccch}
      P. Heymann, H. Garcia-Molina,
      \emph{Collaborative Creation of Communal Hierarchical Taxonomies in
      Social Tagging Systems}, 2006.
      \url{http://ilpubs.stanford.edu:8090/775/1/2006-10.pdf}

    \bibitem{hausdorff:sets}
      F. Hausdorff,
      \emph{Set theory},
      Chelsea Publishing Company, 1991.

    \bibitem{huang:tt}
      J. Huang, K. M. Thornton, E. N. Efthimiadis,
      ``Conversational Tagging in Twitter,''
      \emph{Hypertext}, pp. 173-177, 2010.

    \bibitem{jones:folders}
      W. Jones, A. J. Phuwanartnurak, R. Gill, H. Bruce,
      \emph{Don't Take My Folders Away! Organizing Personal Information to Get
      Things Done}, 2005.
      \url{https://dlib.lib.washington.edu/dspace/bitstream/handle/1773/2031/
           Don%27t%20take%20my%20folders%20away%2c%20current.pdf?sequence=2}

    \bibitem{mathes:folk}
      A. Mathes,
      ``Folksonomies -- Cooperative Classification and Communication Through
      Shared Metadata,''
      \emph{Computer Mediated Communication, LIS590CMC}, 2004.

    \bibitem{shirky:ontology}
      C. Shirky,
      \emph{Ontology is Overrated: Categories, Links, and Tags}, 2005.
      \url{http://www.shirky.com/writings/ontology_overrated.html}

\end{thebibliography}

\end{document}
